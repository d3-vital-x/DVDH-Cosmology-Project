\section{PPN Analysis for DVDH}
\label{sec:ppn_full}

%%%%%%%%%%%%%%%%%%%%%%%%%%%%%%%%%%%%%%%%%%%%%%%%%%%%%%%%%%%%
% SECTION A: Symbolic PPN Formulation
%%%%%%%%%%%%%%%%%%%%%%%%%%%%%%%%%%%%%%%%%%%%%%%%%%%%%%%%%%%%
\subsection{A. Symbolic Structure of PPN Expansion}

In the weak-field regime we expand the metric as:
\[
g_{\mu\nu} = \eta_{\mu\nu} + h_{\mu\nu}, \qquad |h_{\mu\nu}| \ll 1.
\]

The Newtonian potential is defined from:
\[
g_{00} = -1 + 2\Phi_{\rm eff} + \mathcal{O}(h^2).
\]

The PPN parameters \(\gamma\) and \(\beta\) measure:
\begin{itemize}
    \item curvature of space per unit mass (\(\gamma\))
    \item nonlinearity in gravitational superposition (\(\beta\))
\end{itemize}

Symbolic expressions:
\[
\gamma = \frac{1 + \alpha_m(\phi_0)^2}{1 - \alpha_G(\phi_0)^2}
\]
\[
\beta = 1 + \frac{1}{2}F(\alpha_G,\alpha_m,V''(\phi_0)).
\]

%%%%%%%%%%%%%%%%%%%%%%%%%%%%%%%%%%%%%%%%%%%%%%%%%%%%%%%%%%%%
% SECTION B: Explicit DVDH Parameters
%%%%%%%%%%%%%%%%%%%%%%%%%%%%%%%%%%%%%%%%%%%%%%%%%%%%%%%%%%%%
\subsection{B. Explicit DVDH Algebraic Form}

\paragraph{1. Effective Couplings at Background Field \(\phi_0\):}

\[
\alpha_G(\phi_0) = 
\frac{\xi_R \phi_0/M^2}{1 + \xi_R \phi_0^2/M^2},
\qquad
\alpha_m(\phi_0) = \frac{\beta}{M}.
\]

\paragraph{2. PPN Parameter \(\gamma\):}

\[
\gamma - 1 \approx 
2\left(\alpha_G + \alpha_m\right)^2
\]

Explicit DVDH form:
\[
\gamma - 1 \approx 
2\left(
\frac{\xi_R \phi_0/M^2}{1 + \xi_R \phi_0^2/M^2}
+
\frac{\beta}{M}
\right)^2 .
\]

\paragraph{3. PPN Parameter \(\beta\):}

\[
\beta - 1 \approx 
\frac{\alpha_{\rm eff}^2}{4}
-
\frac{1}{4}
\left(\frac{\alpha_{\rm eff}}{\kappa}\right)
\left(\frac{V''(\phi_0)}{\rho_m}\right),
\]

\[
V''(\phi_0) = m_\phi^2 + 3\lambda\phi_0^2
+ \left.\frac{\partial^2 V_{\rm inst}}{\partial\phi^2}\right|_{\phi_0}.
\]

%%%%%%%%%%%%%%%%%%%%%%%%%%%%%%%%%%%%%%%%%%%%%%%%%%%%%%%%%%%%
% SECTION C: Observational Constraints (Cassini)
%%%%%%%%%%%%%%%%%%%%%%%%%%%%%%%%%%%%%%%%%%%%%%%%%%%%%%%%%%%%
\subsection{C. Observational Constraint}

Cassini time-delay experiment gives:

\[
|\gamma - 1| \lesssim 2\times 10^{-5}.
\]

Thus, the combined DVF constraint is:

\[
\left|
\xi_R \frac{\phi_0}{M^2}
+ \frac{\beta}{M}
\right|
\lesssim 3.16\times 10^{-3}.
\]

This ensures that the DVF scalar field does not mediate any detectable fifth force.

%%%%%%%%%%%%%%%%%%%%%%%%%%%%%%%%%%%%%%%%%%%%%%%%%%%%%%%%%%%%
% SECTION D: Next Steps for CLASS/Cobaya
%%%%%%%%%%%%%%%%%%%%%%%%%%%%%%%%%%%%%%%%%%%%%%%%%%%%%%%%%%%%
\subsection{D. Next Steps (For Cosmological Integration)}

\begin{enumerate}
    \item Linearize full field equations from 
    \texttt{field\_equations.tex}.
    \item Solve \(h_{00}\) to extract \(\Phi_{\rm eff}\).
    \item Substitute into \(\gamma,\beta\) formulae.
    \item Implement numerical scan to generate:
    \begin{itemize}
        \item constraint plot \(\gamma-1\) vs \((\xi_R,\beta,\phi_0)\)
        \item allowed parameter contour
    \end{itemize}
    \item Export results into:
    \texttt{07\_RESULTS/ppn\_scan.pdf}.
\end{enumerate}

%%%%%%%%%%%%%%%%%%%%%%%%%%%%%%%%%%%%%%%%%%%%%%%%%%%%%%%%%%%%
% END OF FILE
%%%%%%%%%%%%%%%%%%%%%%%%%%%%%%%%%%%%%%%%%%%%%%%%%%%%%%%%%%%%
