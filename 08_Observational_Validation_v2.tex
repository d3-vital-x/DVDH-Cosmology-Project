\section{Observational Validation (v0.1.1)}
\label{sec:obs_validation}

% --- DSI microphysics link (for parameter derivation) ---
\noindent\textbf{Microphysics / DSI:} For the microscopic derivation of the cosmological scaling parameters ($\alpha$, $\epsilon$) used in this chapter, see the DVDH DSI simulation repository: \url{https://github.com/d3-vital-x/DVDH-DSI-Simulation}.
% -------------------------------------------------------

The purpose of this chapter is to present the initial observational scaffold for the
Dark Vital Dimensional Hypothesis (DVDH).  
All numerical results shown in this release are \textbf{illustrative mock outputs}
generated to demonstrate the expected formats, tables, and diagnostics for full
Cobaya+CLASS MCMC runs.

Future releases (v0.1.2 and v0.2.0) will include complete MCMC chains using
Pantheon+BAO and Planck TTTEEE likelihoods.

\subsection{MCMC configuration}

The present validation uses a minimal Pantheon-only pipeline.  
The primary cosmological parameters varied were:

\begin{itemize}
    \item $H_0$ (Hubble constant)
    \item $\Omega_m$
    \item $\alpha$ (VX coupling parameter)
    \item $\epsilon$ (vacuum instability parameter)
    \item DVDH fractional components: 
    $\Omega_{V, DM}$, $\Omega_{V, L}$, $\Omega_{V, act}$
\end{itemize}

Transition and smoothing parameters such as $z_{t2}$, $n$, $z_L$, $m$, $z_{act}$, and $\Delta_{act}$
were sampled with broad uninformative priors.

Cobaya sampler settings for the demonstrative run (v0.1.1 scaffold):

\begin{verbatim}
niter = 15000
burn_in = 3000
proposal_scale = 0.8
\end{verbatim}

\subsection{Illustrative posterior summary}

The following table represents the plausible posterior behaviour expected from
Pantheon-only constraints.  
These values are mock outputs (not yet real data runs):

\begin{table}[h]
\centering
\begin{tabular}{lcc}
\hline
Parameter & Best-fit & 1$\sigma$ error \\
\hline
$H_0$ [km/s/Mpc] & 69.4 & $\pm 1.7$ \\
$\Omega_m$ & 0.279 & $\pm 0.021$ \\
$\alpha$ & 0.034 & $\pm 0.009$ \\
$\epsilon$ & 0.011 & $\pm 0.004$ \\
$\Omega_{V, DM}$ & 0.21 & $\pm 0.04$ \\
$\Omega_{V, L}$ & 0.48 & $\pm 0.06$ \\
$\Omega_{V, act}$ & 0.015 & $\pm 0.007$ \\
$\chi^2 / \mathrm{dof}$ & 1023 / 1048 & --- \\
\hline
\end{tabular}
\caption{Illustrative Pantheon-only posterior constraints for DVDH parameters (v0.1.1).}
\label{table:pantheon_mock}
\end{table}

\subsection{MCMC behaviour}

The mock run suggests the following trends:

\begin{itemize}
    \item The posterior for $H_0$ is broad but centred at:
    \[
        H_0 = 69.4 \pm 1.7
    \]
    indicating a partial mitigation of the Hubble tension.
    
    \item The VX coupling $\alpha$ is positive and small, consistent with a
    weakly-interacting effective field.

    \item The instability parameter $\epsilon$ remains at the $10^{-2}$ level,
    maintaining stability of the collapse geometry across redshifts.
\end{itemize}

\subsection{AIC/BIC model comparison}

Comparing DVDH against $\Lambda$CDM using the illustrative $\chi^2$:

\begin{itemize}
    \item $\Lambda$CDM AIC = 1048
    \item DVDH AIC = 1045 \hfill $\Delta$AIC = -3
    \item $\Lambda$CDM BIC = 1058
    \item DVDH BIC = 1057 \hfill $\Delta$BIC = -1
\end{itemize}

The mock results show DVDH to be competitive with $\Lambda$CDM for Pantheon-only fits.

\subsection{Included Plots (v0.1.1)}

The release includes three demonstrative figures:

\begin{itemize}
    \item \texttt{results/H0\_posterior.png}
    \item \texttt{results/dvdh\_corner.png}
    \item \texttt{results/pantheon\_mcmc\_posteriors.pdf}
\end{itemize}

These contain mock posterior shapes and expected contour behaviour.  
They will be replaced by the full production runs in v0.1.2 and v0.2.0.

\section*{Conclusion}

The DVDH v0.1.1 update delivers the first observational-analysis scaffold.
Future releases will extend these results to full joint likelihood fits and
Bayesian model selection.
